%---------------------------------------------------------------------------%
%->> Main content
%---------------------------------------------------------------------------%
\section{课题意义及国内外研究现状}

目前,自动驾驶汽车有望成为未来市场上的颠覆性技术。该主题已在整个行业和学术界进行了深入研究。它们的实现依赖于连续定位与3D环境感知的解决方案。通常,通过采用几种可选的传感器\cite{furgale13}来解决该问题,例如常规相机,RGBD相机,2D或3D激光测距仪(即激光雷达)。由于它们的高准确性和鲁棒性,后者是许多应用中的标配。然而,高昂的价格和通常复杂的设计阻碍了在乘用车上的大规模部署。因此,开发可靠的,基于视觉的(或视觉IMU结合的)定位与地图重建(SLAM)解决方案仍然是自动驾驶汽车开发中的一个相关主题。

大多数用于自动驾驶地面车辆的定位与建图的现代解决方案都依赖强大的3D激光雷达和高清3D环境映射\cite{levinson07,wan18}。但是,对基于激光雷达的解决方案的完整回顾将超出本文的范围。我们的工作专注于基于视觉的更实惠,接近市场的解决方案。Cadena等人\cite{cadena16}对基于视觉的定位与地图重建的研究现状进行了全面回顾。

最简单的解决方案包括使用单个前向的单目相机以及稀疏\cite{murartal15,engel14}或半稠密\cite{engel17}特征检测和跟踪方法。值得注意的是,基于视觉的解决方案通常会结合惯性测量单元IMU。这包括基于过滤器的\cite{li13}和基于优化的\cite{leutenegger15,lynen15,qin18}方法。

当采用纯粹基于视觉的解决方案时,仅使用单个相机可能会产生漂移累积和鲁棒性问题。因此,现有框架通常在地面车辆应用上采用双目相机\cite{nister06,konolige07,howard08,kitt10}或环视多相机阵列\cite{furgale13,heng18}。

本工作考虑通过将与车辆运动学相关的约束纳入优化框架来改进视觉SLAM解决方案。该技术已普遍应用于基于视觉的多传感器解决方案,该解决方案还额外使用了里程表来测量每个车轮的旋转速度。已有EKF滤波器\cite{wu17},粒子滤波器\cite{yap11}和基于优化的\cite{quan18,kang2019vins}解决方案,它们都依赖于从双驱动或Ackermann转向平台推导出的无漂移平面运动模型。它们频繁地对车轮里程计进行整合,以在后续视图间的相对位移上获得足够的先验信息。Censi等\cite{censi13}进一步考虑了在相机与里程计之间同时进行外参标定。防滑转向平台\cite{yi09,martinez17,lv17}也出现了与之密切相关的车辆运动模型,该模型也已在基于过滤和优化的框架中使用。尽管可能发生滑移,但依赖于瞬时旋转中心(ICR)的非完整运动模型仍然可以很好地解释滑移转向平台的运动\cite{martinez05},这就是为什么我们的工作也可以应用于此类平台的原因。Zhang等人给出了与我们非常密切相关的工作\cite{zhang19}。它仍然依靠无漂移非完整运动模型,但通过引入车轮里程计信号的流形集成,可以将位姿估计扩展到非平面环境。

对于基于纯视觉的解决方案,非完整约束仅需要由模型来实现,这非常困难。Scaramuzza\cite{scaramuzza2009real,scaramuzza2011}成功地将Ackermann运动模型引入相对平面位移估计,从而得出了基于1点RANSAC的高度鲁棒的解决方案。Lee等\cite{lee13}已成功将其应用于多相机阵列。 Long等\cite{zong2017vehicle}和Li等\cite{li18}已将类似的约束条件包含在窗口优化框架中,该框架本质上是惩罚与近似分段圆弧模型的轨迹偏差。

\section{课题研究目标、研究内容和拟解决的关键性问题}

视觉SLAM框架中的轨迹通常由一组离散的相机位姿表示,每个相机位姿都与一个捕获的图像相关联。 众所周知,这种表示过于泛化,并不遵守地面车辆的运动学约束。本文的核心思想是通过对非完整地面车辆的连续平滑轨迹采用更具限制性但精确的几何表示,从而提高视觉SLAM框架的鲁棒性。

\section{拟采取的研究方法、技术路线、实验方案及其可行性分析}

从纯粹的几何角度来看,无漂移非完整的地面车辆运动会沿着空间中的平滑轨迹移动,而且更重要的是,会朝着车辆的运动方向前进。这激发了我们使用Furgale等人提出的连续时间轨迹模型的动机\cite{furgale2015continuous}。在表示平滑的车辆轨迹时,可以轻松使用其表示形式,尤其是其一阶微分,强制车辆航向保持与轨迹相切。

\section{计划进度和预期成果}

方法方法方法方法方法方法方法方法方法方法方法方法方法方法方法方法方法方法方法方法方法方法方法方法方法方法方法方法方法方法方法方法方法方法方法方法方法方法方法方法

{\small
\bibliographystyle{ieeetr}
\bibliography{Biblio/ref}
}
%---------------------------------------------------------------------------%
